\documentclass{beamer}

\usetheme{default} % change these to dramatically change the styling
\usecolortheme{dove} % this one too
\setbeamertemplate{navigation symbols}{} % don't use navigation tools on slides
%\setbeamertemplate{footline}{\scriptsize{\hfill\insertframenumber}\hspace*{5pt}\vspace*{5pt}} 

\title{Helicopter Workshop}
\author{Andrew Barry | Computer Science and Artificial Intelligence Laboratory | Massachusetts Institute of Technology}
\date{}

\usepackage{textcomp}
\usepackage{multimedia}
\usepackage[absolute,overlay]{textpos}
\usepackage{xcolor}
\usepackage{textcomp}
\usepackage[strict]{changepage}
\usepackage{biblatex}
\usepackage{subfigure}
\usepackage{mathtools}

\usepackage[latin1]{inputenc}
\usepackage{tikz}
\usetikzlibrary{shapes,arrows}

\usepackage{amsmath}
\usepackage{amssymb}

\usepackage{ulem}
\usepackage{extrabeamercmds}


%  \textcircled{c} 

%\usebackgroundtemplate{%
%    \vbox to \paperheight{\hbox to \paperwidth{\hfil\includegraphics[width=1.1in]{../figures/simulatorGridBackground.pdf}}\vfil}
%}

% --- uncomment for black background ---
%\setbeamertemplate{background canvas}[vertical shading][bottom=black,top=black]
\setbeamercolor{background canvas}{bg=black}
\setbeamercolor{normal text}{bg=black,fg=white}
\setbeamercolor{structure}{bg=black, fg=white}
\setbeamercolor{title}{bg=black, fg=white}
% --------------------------------------

\bibliography{elib,paper-summaries}

\setcounter{MaxMatrixCols}{20}
\newcommand{\yaw}{\psi}
\newcommand{\pitch}{\theta}
\newcommand{\roll}{\phi}


% Define block styles
\tikzstyle{decision} = [diamond, draw, fill=blue!20, 
    text width=4em, text badly centered, node distance=3cm, inner sep=0pt]
    
\tikzstyle{block} = [rectangle, draw, fill=white, 
    text width=5em, text centered, rounded corners, minimum height=3em]
    
\tikzstyle{blockgray} = [rectangle, draw, fill=gray, 
    text width=5em, text centered, rounded corners, minimum height=3em]
    
\tikzstyle{blockhighlight} = [rectangle, draw, fill=blue!15,
    text width=5em, text centered, rounded corners, minimum height=3em]
    
\tikzstyle{blockhighlight2} = [rectangle, draw, fill=red!15,
    text width=5em, text centered, rounded corners, minimum height=3em]
    
\tikzstyle{blocklonghighlight2} = [rectangle, draw, fill=red!15,
    text width=7em, text centered, rounded corners, minimum height=3em]
    
\tikzstyle{blocklong} = [rectangle, draw, fill=white, 
    text width=6em, text centered, rounded corners, minimum height=3em]
    
\tikzstyle{blocklonglong} = [rectangle, draw, fill=white, 
    text width=8em, text centered, rounded corners, minimum height=3em]
    
\tikzstyle{line} = [draw, -latex']

\tikzstyle{dashedline} = [draw, -latex', dashed]

\tikzstyle{cloud} = [ellipse, draw, fill=white, 
    text width=5em, text centered, rounded corners, minimum height=3em]
    
\tikzstyle{cloudsmall} = [ellipse, draw, fill=white, 
    text width=4em, text centered, rounded corners, minimum height=3em]%[draw, ellipse,fill=red!20, node distance=3cm, minimum height=2em]
    
\tikzstyle{blockhidden} = [rectangle, draw=white, fill=white, 
    text width=5em, text centered, rounded corners, minimum height=3em, text=white]

\tikzstyle{cross} = [cross/.style={path picture={ 
  \draw[black](path picture bounding box.south east) -- (path picture bounding box.north west) (path picture bounding box.south west) -- (path picture bounding box.north east);
}}]
    
    
\begin{document}

\startframe

% what's in this talk
%   welcome
%   
%   what we're doing
%      helicopters
%      circuits
%   how to start
%       what a breadboard is


\fullFrameImage{figures/helicopter/S107G-large.jpg}{
    \vspace{0.85\textheight}
    \hspace{-22pt}
    \textcolor{white}{Andrew Barry} \\
    \hspace{-17pt}\textcolor{white}{MIT Ph.D. Candidate in Computer Science}
    \vspace{220pt}
    \CopyrightText{Image \textcircled{c} 2011, William Warby}
}

\frame{
    Today you will:
    
    \begin{itemize}\addtolength{\itemsep}{0.5\baselineskip}
        \item<2-> Learn to fly micro-helicopters
        \item<3-> Build a circuit to control the helicopters
        \item<4-> Program autonomous flight maneuvers
        
    \end{itemize}
}


\frame{
    
    {\Large What if I crash?}
    
    \vspace{20pt}
    
    \onslide<2->{\hspace{20pt} \textbf{\LARGE Trust me, you will.}}
    
}

\fullFrameImageScaled{figures/glamor/fullPlane.jpg}{}


\fullFrameMovieAvi[noloop]{videos/crashes/full-crashes-prophang-just-one-c}

\fullFrameMovieAvi[noloop]{videos/crashes/full-crashes-prophang-just-one-c}

\fullFrameMovieAvi{videos/crashes/full-crashes-prophang-all-but-first-c}

\fullFrameMovieAvi[noloop]{videos/prophang/prophang_LQRe5}

%\fullFrameMovieAvi[noloop]{videos/crashes/deltawing/HUD-flip-1x-crash-c}

%\fullFrameMovieAvi[noloop]{videos/crashes/jun02-horz9-cut-c}

%\fullFrameMovieAvi[noloop]{videos/knife-edge/knife-edge-horz-cut-c}


\frame{
    Before we start ---
    \onslide<2->
    {
        \vspace{20pt}
        
        You'll build your circuit on a breadboard:
        
        \vspace{10pt}
        
        \includegraphics[width=\textwidth]{figures/parts/breadboard-small.png}
    }
}


\blackframe


%\setbeamertemplate{bibliography item}[text]
%\bibliographystyle{unsrt}

%\bibliography{../PowersMarraBarryGaitDetection}    
%\bibliographystyle{abbrv}

    
\end{document}


